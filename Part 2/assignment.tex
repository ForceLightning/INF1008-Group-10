\documentclass{report}

\input{../preamble}
\input{../macros}
\input{../letterfonts}

% \title{\Huge{INF1008}\\Data Structures \& Algorithms\\\Large{Part 1 Assignment}}
% \author{\huge{Group 10}}
% \date{}

\begin{document}

% \maketitle
\begin{titlepage}
	\centering
	\includegraphics[width=0.8\textwidth]{sit-logo-primary}\par\vspace{1cm}
	{\scshape\Huge{INF1008}\\Data Structures \& Algorithms\\\Large{Part 1 Assignment}\par}
	\vspace{1cm}
	{\huge{Group 10}\par}
	\vspace{0.5cm}
	\begin{table}[H]
		\large
		\centering\begin{tabular}{ll}
			\toprule
			Name & Student ID\\
			\midrule
			Wong Yok Hung & 2202391\\
			Christopher Kok & 2203503\\
			Chong Hou Wei & 2201783\\
			Vivian Ng & 2203557\\
			Methinee Ang & 2202781\\
			\bottomrule
		\end{tabular}
	\end{table}
	\vfill
\end{titlepage}
\newpage% or \cleardoublepage
% \pdfbookmark[<level>]{<title>}{<dest>}
\pdfbookmark[section]{\contentsname}{toc}
\tableofcontents
\listoffigures
\pagebreak
\chapter*{The Problem}
US based phone numbers have been extracted from a bunch of documents and are available in a file, one phone number per line. Since people have many ways of writing phone numbers, the format of the numbers varies quite a bit. Numbers are always a minimum of 10 digits (3-digit area code and 7-digit number) but among the key variations, some numbers come with a ``+1'' prefix (country code for US is 1), other times there is only a ``1'' (as in ``1800'' numbers), the usual blocks are separated by either a space or a dash, the area code is sometimes within braces, sometimes all the digits are contiguous (no spaces), \emph{etc}. Here are some key formats:
\begin{itemize}
	\item +1-732-732-5555
	\item 1-732-732-5555
	\item (732) 732-5555
	\item 7327325555
	\item 732 7325555
\end{itemize}
For simplicity you can assume that the area code can be any 3 digit string that begins with a non-zero digit. You can also assume that the 7-digit number at the end may begin with a 0. So, for example the string ``100 0000000'' will be accepted as a phone number (for the purpose of this exercise!) Duplicates are allowed, in the same or different format.
\chapter{}
\section{Problem Statement}
Your first task is to find the median 10-digit phone number in a given input list (on the command-line). If there are an even number of phone-numbers, \(2n\), then the \(n\)th and \((n+1)\)th number from the sorted list must be output. For the purposes of this exercise, you should simply print out the median(s) as a 10-digit string, without any spaces. If there are 2 medians, they should simply be printed on the same line with comma separating them. Duplicates must be preserved. For example, in the list of 5 numbers above, the single median would be 7327325555 and so the output would be:\\7327325555\\If the input list were just the first 4 numbers, then there would be 2 medians, identical in this case, so the output would be:\\7327325555,7327325555\\
\chapter{}
\section{Problem Statement}
2.	Your second task is to extend the same program by allowing it to take TWO additional command-line inputs: a 10-digit number (that starts with a non-zero digit) and a \(K\). If these additional inputs are provided, the program should print out the \(K\) unique numbers nearest (in terms of numerical difference) to the given 10-digit number. Duplicate numbers are considered ties and should also be printed. Thus, for example, if the program executable is called `findnumbers' and the list of numbers above are in a file called `phonescraped', then following is the command-line invocation for a number with \(k=1\): 
|Findnumbers phonescraped 7327325550 1| For this, the output would be:\\
7327325555\\
7327325555\\
7327325555\\
7327325555\\
7327325555\\
Since all numbers are tied and the closest to the input number.\\
Try to make the search as efficient as possible. As this will be part of the assessment.

\end{document}
